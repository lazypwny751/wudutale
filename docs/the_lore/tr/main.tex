\documentclass[12pt]{article}
\usepackage[utf8]{inputenc}
\usepackage{graphicx}
\usepackage{hyperref}
\usepackage{geometry}

\geometry{
 a4paper,
 total={170mm,257mm},
 left=20mm,
 top=20mm,
}

\usepackage[turkish]{babel}

\title{\textbf{Gorkitale: Hakikat}}
\author{ByCh4n Group}
\date{\today}

\begin{document}

\maketitle

\section*{Mukaddime}

\begin{center}
    \textit{"Gök kubbe şahit olsun ki, hakikat yerin yedi kat dibinde de olsa, biz onu bulup tebliğ edeceğiz."}
\end{center}

\vspace{1cm}

Bak güzel kardeşim, bu okuyacağın sadece bir oyunun hikayesi değil, bir hidayet yolculuğudur. Biz \textbf{Ademoğlu}(Chara)'yuz; ama o bildiğin, nefsine yenik düşmüş, dünyalık peşinde koşan sıradan bir beşer değil. Biz hidayete ermiş, nur yüzlü, kalbi selim bir Ademoğlu'yuz(Chara). Agartha'nın karanlık dehlizlerinde bir ışık, bir umut kapısıyız.

\section*{Evvel Zaman: Mahluklar ve Ademoğlu}

Vaktiyle, dünya henüz fitneyle tanışmamışken, mahluklar ve ademoğlu bir arada, huzur içinde yaşardı. Ne bir kavga vardı, ne de bir nifak. Mahluklar kendi hallerinde, insanlar kendi işlerinde; rızıklarını paylaşır, aynı güneşin altında ısınır, aynı ayın altında dinlenirlerdi. O zamanlar dünya, adeta bir gül bahçesiydi.

\section*{Fitne ve Tekfir: Hariciler ve Selefiler}

Lakin her güzel rüyanın bir sonu olduğu gibi, bu huzur iklimi de bir gün bozuldu. Toplumun içinden çıkan bir grup \textbf{Harici ve Selefi}, mahlukları hedef aldı. "Bunlar bizden değil, bunlar yoldan çıkmış, bunlar bid'at ehli!" diyerek mahlukları tekfir ettiler. Bu fitnenin asıl sebebi, tarihin tozlu sayfalarında kaybolup gitti (burası kayıp bir vakadır, topluluk bu karanlık noktanın sebebini kendi ferasetiyle bulsun). 

Bu haksız tekfir karşısında mahluklar, sabırlarını yitirip taşkınlık çıkarmaya başladılar. Yeryüzünde büyük bir kaos baş gösterdi; ne nizam kaldı, ne de intizam.

\section*{Agartha'ya Hicret: Yerin Dibine Hapis}

İnsanoğlu, bu büyük taşkınlığı durdurmak ve dünyayı eski huzuruna kavuşturmak için mahlukları yerin en dibine, Agartha denilen o karanlık diyara hapsetti. Aslında Agartha, insanların mahlukları diri diri gömdükleri, üzerlerine mühürler vurdukları devasa bir kabristandı. Kapılar mühürlendi, tılsımlar okundu. Mahluklar unutuldu, insanlar ise yüzeyde kendi dünyalarını kurdular. Lakin unutulan her şey, bir gün hatırlanmak üzere bekler.

\section*{Ademoğlu'nun İmtihanı ve Büyük Beddua}

Gelelim asıl meseleye hafız... Bizim Ademoğlu, bir gün öyle bir bedduaya, öyle ağır bir büyüye maruz kaldı ki, tabiri caizse ebesininkini tersten gördü. Kalbi daraldı, ruhu sıkıştı, dünya başına dar geldi. Yeryüzünde çalmadık kapı, sormadık alim bırakmadı ama nafile; dermanı kimse bulamadı. Sonunda bir ihtiyar ona fısıldadı: "Senin derdinin dermanı, insanların unuttuğu, yerin yedi kat dibine gömdüğü o kadim varlıklarda, yani Agarthalı agalardadır."

Ademoğlu, korkuyla karışık bir ümitle Agartha'nın mühürlü kapısına vardığında karşısına heybetli Agartha Gardiyanı çıktı. Gardiyan, Ademoğlu'nun perişan halini görünce merhamet etti ve ona bembeyaz bir şişe uzattı: "Bu beyaz mahlukat (monster) içeceğidir. Bunu iç ki ruhun bu karanlığa alışsın, bedenin bu ağırlığı kaldırsın. İç ve içeri gir," dedi. Ademoğlu o iksiri içtiği an, damarlarında akan kanın rengi değişti ve Agartha'nın rutubetli, karanlık dehlizlerine ilk adımını attı.

\section*{Günümüz ve Dava: Ya Tebliğ Ya Tekfir}

İşte olay tam burada başlıyor hafız! Biz, o beyaz iksiri içip Agartha'nın unutulmuş koridorlarına indik. Karşımıza çıkanlar kimler? Sans'mış, Rarity'ymiş, Eilish'miş... Hepsi birer zındık olmuş çıkmış. Sans tembellikten başını kaldıramıyor (tembellik haram!), Rarity dünya malına tamah etmiş, Eilish ise boşlukta kendi kendine konuşuyor, belli ki cinlenmiş garibim.

Lakin bizim işimiz sadece kılıç sallamak değil. Bizim silahımız kelam, zırhımız ise imandır. Agartha'nın her bir köşesinde, her bir "boss" savaşında asıl mesele bilek gücü değil, yürek gücüdür. Karşımıza çıkan mahluklara önce güzellikle anlatacağız, "Gelin bu yoldan dönün, Agartha'yı bir ilim irfan yuvasına çevirelim, bu karanlığı nurla aydınlatalım" diyeceğiz. Hidayete ererlerse ne ala, omuz omuza veririz. Ama inat ederlerse, o zaman basıyoruz tekfiri, geçiyoruz. 

Asıl mesele kalplerin paniğidir. Glitch dediğin şeytanın vesvesesidir; kodlardaki hata değil, ruhlardaki boşluktur. Biz bu boşluğu doldurmaya, Agartha'nın o karanlık kabristanını bir medreseye çevirmeye geldik. Biiznillah, bu dava yarım kalmayacak!

\section*{Yusuf ordularının işgali}

Yusuf uma ordusu eğiterek agarthayı işgale gelir.

\end{document}
